\addchap{Acknowledgments}
\begin{refsection}

This book is based on different papers, discussions, and courses we have written and conducted in recent years. Particularly, we want to highlight the Digiling Online Course we developed during the Erasmus+ sponsored project phase during 2016 and 2019 (see \href{www.digiling.eu}{Website of the project}). 
The idea of the project was to originate training in Digital Linguistics. A needs analysis among market players showed a growing demand for employees with training in this field, but no European University offered a programme in Digital Linguistics. These were the goals:\footnote{Learn more on \href{http://www.digiling.eu/about/}{the project's website}.}
 \begin{itemize}
   \item Find out about the market's needs (survey with market players) .
   \item Create an internationally approved model curriculum for Digital Linguistics (combining old and new courses).
   \item Train the trainers in designing online courses.
   \item Design online courses for selected modules (all were evaluated, localised and implemented into the online learning environment).
   \item Disseminate and sustain the project outcomes.
\end{itemize}
The project targets students, teachers, researchers, and other actors at universities, companies, organisations, public institutions and other users of digital language services. You can find further exercises concerning post-editing or other exciting courses on digital linguistics, like localisation, term mining and managing or programming in Python on our \href{https://learn.digiling.eu}{project online platform}.


\printbibliography[heading=subbibliography]
\end{refsection}